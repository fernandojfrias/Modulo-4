\documentclass[12pt]{article}
\usepackage[utf8]{inputenc}
\usepackage[spanish]{babel}
\usepackage{geometry}
\usepackage{graphicx}
\usepackage{hyperref}
\usepackage{titlesec}
\usepackage{setspace}
\usepackage{amsmath}
\usepackage{lmodern}
\usepackage{fancyhdr}
\usepackage{enumitem}

\geometry{a4paper, margin=2.5cm}
\setstretch{1.5}
\titleformat{\section}{\normalfont\Large\bfseries}{\thesection}{1em}{}
\pagestyle{fancy}
\fancyhf{}
\rhead{\thepage}
\lhead{India en la economía mundial}

\title{\textbf{El ascenso de India en la economía mundial: \\Una perspectiva estratégica desde las macrotendencias globales}}
\author{Lucas Fernandez, Fernando Frías, Martiniano Giménez, Tomás Manganelli}
\date{Junio 2025}

\begin{document}

\maketitle

\section*{Abstract}

India está emergiendo como una de las principales potencias económicas del siglo XXI. Este informe analiza los factores detrás de su ascenso en el contexto de las macrotendencias globales: reorganización de cadenas de valor, revolución digital, transición energética y transición hegemónica. Se presentan las ventajas estructurales de India, sus políticas de desarrollo, sus desafíos internos y su creciente rol en la geopolítica internacional. Se concluye que India está configurando un nuevo eje económico y estratégico que redefine el orden multipolar.

\section{Introducción: Macrotendencias y reorganización global}

Las macrotendencias son procesos estructurales de largo plazo que moldean el escenario global. Entre las más relevantes figuran:

\begin{itemize}
  \item Revolución tecnológica y digitalización
  \item Cambio climático y transición energética
  \item Reconfiguración geopolítica con declive relativo de Occidente
  \item Crisis de la globalización tradicional y surgimiento de bloques regionales
\end{itemize}

El documento base del Seminario de Macrotendencias conceptualiza este contexto como un “interregno” global. India emerge en este escenario como un actor bisagra entre el Norte y el Sur, entre tecnología y manufactura, entre tradición y modernidad.

\section{Crecimiento económico de India: panorama actual}

India es actualmente la quinta economía mundial en términos nominales y la tercera por paridad de poder adquisitivo (PPA). En 2024 creció un 6.8\%, liderando el G20, y se proyecta que en 2025 alcanzará un crecimiento del 7\%.

Este dinamismo se basa en:

\begin{enumerate}[label=\alph*)]
  \item Mercado interno robusto y joven (edad media: 28 años)
  \item Inversión en infraestructura digital y física
  \item Reformas estructurales en banca, comercio e inversión
  \item Diversificación productiva con orientación exportadora
\end{enumerate}

\section{India y el cambio en las cadenas globales de valor}

La pandemia de COVID-19 aceleró la necesidad de diversificar las cadenas productivas fuera de China. India se posiciona como destino clave del \textit{China+1 strategy} de muchas multinacionales.

\subsection*{Make in India y producción avanzada}

Lanzado en 2014, el programa \textbf{Make in India} impulsa la manufactura nacional. Entre sus logros destacan:

\begin{itemize}
  \item Creación de Zonas Económicas Especiales (ZEE)
  \item Incentivos PLI (Production Linked Incentive)
  \item Aumento de la inversión extranjera directa (IED): en 2024 superó los USD 85.000 millones
\end{itemize}

Empresas como Apple, Tesla, Samsung y Foxconn ya han instalado plantas de ensamblaje en India.

\section{El rol estratégico de India en la economía digital}

India es una potencia digital emergente. Su ecosistema de \textit{startups}, el segundo más grande del mundo, está creciendo a tasas aceleradas. Con más de 800 millones de usuarios de Internet, ha logrado combinar escala, acceso y costos bajos.

\subsection*{India Stack: inclusión financiera y digitalización}

La plataforma India Stack, compuesta por Aadhaar (identidad digital), UPI (pagos) y DigiLocker (documentación), ha revolucionado el acceso a servicios:

\begin{itemize}
  \item Más de 400 millones de cuentas bancarias abiertas
  \item 90\% de transacciones electrónicas usando UPI
  \item Digitalización de servicios públicos y privados
\end{itemize}

\section{Comparativa estratégica: India vs. China}

Aunque a menudo se compara con China, India sigue un camino distinto:

\begin{tabular}{|p{5cm}|p{5cm}|p{5cm}|}
\hline
\textbf{Dimensión} & \textbf{China} & \textbf{India} \\
\hline
Modelo político & Partido único & Democracia multipartidaria \\
\hline
Demografía & Envejecimiento rápido & Bono demográfico activo \\
\hline
Política industrial & Centralizada y planificada & Mixta y orientada a mercados \\
\hline
Rol internacional & Expansivo, confrontativo & Equilibrado, no alineado \\
\hline
\end{tabular}

Ambas potencias lideran el ascenso del Sur Global, pero India es percibida como un socio más predecible por Occidente.

\section{India y el Sur Global: liderazgo emergente}

India busca liderar un bloque de países en desarrollo mediante:

\begin{itemize}
  \item La Alianza Solar Internacional
  \item Cumbre Voz del Sur Global (2023)
  \item Cooperación tecnológica con África y América Latina
  \item Participación en el G20, BRICS, y Quad
\end{itemize}

India articula una agenda multilateral basada en el desarrollo inclusivo, la resiliencia climática y la equidad digital.

\section{Desafíos estructurales internos}

India aún enfrenta profundas barreras internas:

\begin{itemize}
  \item \textbf{Desigualdad regional}: Estados como Bihar o Uttar Pradesh tienen indicadores sociales muy por debajo del promedio nacional.
  \item \textbf{Educación}: Si bien la matrícula escolar ha aumentado, la calidad sigue siendo dispareja.
  \item \textbf{Salud}: India destina menos del 2\% del PIB a salud pública.
  \item \textbf{Empleo informal}: Casi el 80\% de los trabajadores están fuera del sistema formal.
\end{itemize}

Estos factores pueden limitar el desarrollo a largo plazo si no se abordan con políticas sistémicas.

\section{India y la competitividad sistémica}

Desde el marco de la \textit{competitividad sistémica}, India presenta avances claros:

\begin{description}
  \item[Nivel micro:] Empresas como Tata, Reliance y Mahindra son innovadoras y globales.
  \item[Nivel meso:] Ecosistemas de innovación en Bangalore, Hyderabad y Pune.
  \item[Nivel macro:] Políticas fiscales prudentes y crecimiento sostenido.
  \item[Nivel meta:] Instituciones democráticas estables, aunque con desafíos de gobernanza.
\end{description}

\section{Escenarios prospectivos al 2035}

Se pueden esbozar tres escenarios posibles:

\begin{itemize}
  \item \textbf{Escenario ascendente:} India supera a Alemania y Japón y se convierte en la tercera economía mundial.
  \item \textbf{Escenario intermedio:} Mantiene un crecimiento moderado pero enfrenta cuellos de botella estructurales.
  \item \textbf{Escenario regresivo:} Tensiones sociales o fallas en la inclusión digital frenan su potencial.
\end{itemize}

La evolución dependerá de su capacidad para institucionalizar reformas, integrar desarrollo sostenible y sostener una política exterior autónoma.

\section*{Conclusiones finales}

India es una pieza clave en la nueva arquitectura económica y estratégica global. A diferencia de modelos anteriores basados en imitación, India desarrolla un camino propio, híbrido y adaptativo. Si logra superar sus desafíos internos, tiene el potencial de convertirse no solo en una potencia económica, sino también en un referente de desarrollo inclusivo, sostenible y democrático en el siglo XXI.

\end{document}


